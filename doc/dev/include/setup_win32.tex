\chapter{Setting up your build environment\label{toc:setupbenv}}
This chapter deals with setting up your build environment, i.e. take the necessary step to get ready to build CtP2 from the source.\\
If you have already setup your build environment, you can continue with the next chapter.
\begin{section}{Windows}
\begin{subsection}{System requirements}
Your system is supposed to meet the following system requirements:
\begin{itemize}
\item A PC with a Pentium II 266 MHz processor or higher\\
(or comparable compatible processor (AMD/Cyrix/IBM/Via))
\item At least 64 MB RAM
\item Plenty of space on your hard disc (a few GByte, if you keep backup copies)
\item Windows 98, 98 SE, ME, NT, 2000, XP or later
\item Needed software should be installed in directories with alphanumeric names only (just made of [A-Z,a-z,0-9] and \textbackslash{});
if you already installed everything into different named directories, you might have success using the 8.3-name listed via the dir command in the commandline window or quoting the directories name in \". However, if this doesn't work for you, you'd have to install everything again, sorry.
\end{itemize}
\end{subsection}% System requirements

\begin{subsection}{Install Civilization Call to Power II}
Provided you haven't already installed Windows and Civilization Call to Power II, you need to install them now. Make sure each directoryname of the full installation path only contains letters and numbers.

For Call to Power II, choose a full setup (e.g. to \textit{C:\textbackslash{}Program\_Files\textbackslash{}Activision\textbackslash{}CtP2}). (\textbf{Note the underscore})

Refer to the documentation bundled with your software on how to do this.
\end{subsection}% Install Civilization Call to Power II

\begin{subsection}{Install the CtP2 1.11 patch}
Download and install the CtP2 v1.11 patch from \href{http://apolyton.net/ctp2/}{here}.
\end{subsection}% Install the CtP2 1.11 patch

\begin{subsection}{Backup your CtP2 directory}
Create a copy of your CtP2 installation.
\end{subsection}% Backup your CtP2 directory

\begin{subsection}{Install MS Visual C++ 6.0}
Install MS Visual C++ 6.0, if you did not do so already (e.g. to C:\textbackslash{}MSVC).

Refer to the documentation bundled with your software on how to do this.
Afterwards, download \href{http://msdn.microsoft.com/vstudio/downloads/updates/sp/vs6/sp6/default.aspx}{Visual Studio 6.0 Service Pack 6} and follow the installation instructions on the same link.

Note: Instead of MS Visual C++ 6.0, you may also install the Visual C++ 6.0 Introductory Edition, MS Visual C++ 6.0 Authorenedition or any other localized version of the Introductory Edition. You still find it shipped with some C++-Books, and you may succeed upgrading them with Visual C++ 6.0 SP6.

If you intend to use .NET, you may need to do some porting work (keeping backwards compatibility to Visual Studio 6.0).
\end{subsection}% Install MS Visual C++ 6.0

\begin{subsection}{DirectX SDK Setup}
There are two possible and tested DirectX setups for now.

You can choose between the DirectX SDK 7.0 and DirectX SDK 9.0b. If you choose the first one, you have the same SDK versions that CtP2 was compiled with. This means that additionally you'll have to install the Direct Media SDK 6.0 as well.

If you choose to use DirectX SDK 9.0b, you don't need the Direct Media SDK 6.0, but have to compile the DirectShow BaseClasses shipped with the DirectX SDK yourself.

\begin{subsubsection}{Alternative 1: Using DirectX SDK 7.0}
At the time of writing, DirectX SDK 7.0 has become unavailable. If you have a link, please place a post in the \href{http://apolyton.net/forums/forumdisplay.php?s=&forumid=213}{CtP2-Source Code Project} forum.

Install the DirectX 7.0 SDK. Then, download the \href{http://www.microsoft.com/downloads/details.aspx?FamilyId=FD044A42-9912-42A3-9A9E-D857199F888E&displaylang=en}{Direct Media 6.0 SDK} and follow the installation instructions.

Launch the Visual C++ IDE. Select menu \textit{Tools}, then submenu \textit{options}. Within the upcoming dialog, select tab \textit{Directories}. Select the include directories path, and make sure, these directories are at the top of the path:
\begin{verbatim}
C:\DXSDK\Include
C:\DXMedia\Include
C:\DXMedia\Classes\Base
\end{verbatim}

Then, select the lib directories path, and make sure, these directories are at the top of the path:
\begin{verbatim}
C:\DXSDK\Lib
C:\DXMedia\Lib
\end{verbatim}

Note: If you installed the SDKs to somewhere different, of course the paths C:\textbackslash{}DXSDK and C:\textbackslash{}DXMedia must be replaced with the locations of your installations.

Note: The download link to the DXMedia SDK may become inactive soon, the SDKs are not listed at the official site at \href{http://www.microsoft.com/downloads/details.aspx?FamilyId=FD044A42-9912-42A3-9A9E-D857199F888E&displaylang=en}{DirectX download site}.
\end{subsubsection}% Alternative 1: Using DirectX SDK 7.0
\begin{subsubsection}{Alternative 2: Using the latest DirectX SDK}
At time of writing, the latest DirectX SDK can be downloaded \href{http://www.microsoft.com/downloads/details.aspx?FamilyId=FD044A42-9912-42A3-9A9E-D857199F888E\&displaylang=en}{here}. If the link doesn't work, download the latest DirectX SDK from \href{http://www.microsoft.com/downloads/details.aspx?FamilyId=FD044A42-9912-42A3-9A9E-D857199F888E&displaylang=en}{here}.

Then, follow the instructions on the bottom of the link to install the SDK. Newer SDKs will suggest a path containing whitespace and brackets, so you must choose a different installation path, e.g. C:\textbackslash{}DXSDK.
Make sure you select "Headers and Libraries" and the C++ Samples as well.

Launch the Visual C++ IDE. Select menu \textit{Tools}, then submenu \textit{options}. Within the upcoming dialog, select tab \textit{Directories}. Select the include directories path, and make sure, these directories are at the top of the pathi, if the installer did not already enter them at the correct position:

\begin{verbatim}
C:\DXSDK\Include
\end{verbatim}

Then, select the lib directories path, and make sure, these directories are at the top of the path:

\begin{verbatim}
C:\DXSDK\Lib
\end{verbatim}

Afterwards, open the DirectShow BaseClasses workspace by loading the file\\ \textit{C:\textbackslash{}DXSDK\textbackslash{}Samples\textbackslash{}C++\textbackslash{}DirectShow\textbackslash{}BaseClasses\textbackslash{}baseclasses.dsw}.

Build the release version (strmbase.lib):
Open \textit{Build - Set active configuration} and select \textit{BaseClasses - Win32 Release}. Then run \textit{Build - strmbase.lib}.

Build the debug version (strmbased.lib):
Open \textit{Build - Set active configuration} and select \textit{BaseClasses - Win32 Debug}. Then run \textit{Build - strmbased.lib}.

Again, select menu \textit{Tools}, then submenu \textit{options}. Within the upcoming dialog, select tab \textit{Directories}. Select the include directories path, and make sure, these directories are at the top of the path in this order:

\begin{verbatim}
C:\DXSDK\Samples\C++\DirectShow\BaseClasses
C:\DXSDK\Include
\end{verbatim}

Then, select the lib directories path, and make sure, these directories are at the top of the path, in that order:

\begin{verbatim}
C:\DXSDK\Samples\C++\DirectShow\BaseClasses\Debug
C:\DXSDK\Samples\C++\DirectShow\BaseClasses\Release
C:\DXSDK\Lib
\end{verbatim}
\end{subsubsection}% Alternative 2: Using the latest DirectX SDK
\end{subsection}% DirectX SDK Setup

\begin{subsection}{Optional: Install Simple DirectMedia Layer libraries}
If you want to build CtP2 against \href{http://www.libsdl.org}{Simple DirectMedia Layer}, you also need to install the SDL libraries mentioned below.

\begin{subsubsection}{Install SDL 1.2.7}
Download \href{http://www.libsdl.org/release/SDL-devel-1.2.7-VC6.zip}{SDL-devel-1.2.7-VC6.zip} and unpack it to \textit{C:\textbackslash{}libs}.
If the download doesn't work, proceed so with the latest version from \href{http://www.libsdl.org/download-1.2.php}{http://www.libsdl.org/download-1.2.php}.
\end{subsubsection}% Install SDL 1.2.7

\begin{subsubsection}{Install SDL\_image 1.2.3}
Download \href{http://www.libsdl.org/projects/SDL_image/release/SDL_image-devel-1.2.3-VC6.zip}{SDL\_image-devel-1.2.3-VC6.zip} and unpack it to \textit{C:\textbackslash{}libs}.
If the download doesn't work, proceed so with the latest version from \href{http://www.libsdl.org/projects/SDL_image/}{http://www.libsdl.org/projects/SDL\_image/}.
\end{subsubsection}% Install SDL\_image 1.2.3

\begin{subsubsection}{Install SDL\_mixer 1.2.5a}
Download \href{http://www.libsdl.org/projects/SDL_mixer/release/SDL_mixer-devel-1.2.5a-VC6.zip}{SDL\_mixer-devel-1.2.5a-VC6.zip} and unpack it to \textit{C:\textbackslash{}libs}.
If the download doesn't work, proceed so with the latest version from \href{http://www.libsdl.org/projects/SDL_mixer/}{http://www.libsdl.org/projects/SDL\_mixer/}.
\end{subsubsection}% Install SDL\_mixer 1.2.5a

\begin{subsubsection}{Add the SDL libraries to your Visual Studio library paths}
Launch the Visual C++ IDE. Select menu \textit{Tools}, then submenu \textit{options}. Within the dialog showing up, select tab \textit{Directories}. Select the include directories path, and make sure, these directories are at the top of the path:

\begin{verbatim}
C:\libs\SDL-1.2.7\include
C:\libs\SDL_image-1.2.3\include
C:\libs\SDL_mixer-1.2.5\include
\end{verbatim}

Finally, select the lib directories path, and make sure, these directories are at the top of the path:

\begin{verbatim}
C:\libs\SDL-1.2.7\lib
C:\libs\SDL_image-1.2.3\lib
C:\libs\SDL_mixer-1.2.5\lib
\end{verbatim}
\end{subsubsection}% Add the SDL libraries to your Visual Studio library paths
\end{subsection}% Optional: Install SDL

\begin{subsection}{Add STL fixes (MSCV 6 only)}
If you are running MSVC6, you will need to add ctp2\_code/compiler/msvc6/stlfixes to your include path. This should come before the standard MSVC6 includes but after the DXMedia/DirectX includes.
\end{subsection}

\begin{subsection}{Unpacking the source}
If not already done so, download the source from \href{http://ctp2files.apolyton.net/source/}{apolyton.net}.

Install or unzip the source to a directory of your choice (e.g. C:\textbackslash{}ctp2source).
\end{subsection}% Unpacking the source

\begin{subsection}{Setup your environment variables}
If you run Windows 98, Windows98 SE or Windows ME, edit your autoexec.bat to contain
\begin{verbatim}
SET CDKDIR=C:\ctp2source\bin
SET PATH=%PATH%;%CDKDIR%
\end{verbatim}
You must make sure that the CDKDIR variable points to the subdirectory bin, where some programs like \textit{byacc.exe} and \textit{flex.exe} exist. If you have spaces or special chars in this path, you might have luck quoting the path with \", but if compiling ctp2 fails later on with messages that files or commands cannot be found, it's most likely you'll have to install to a different path again.

If you run Windows NT, 2000, XP or later, edit your environment variables to contain those paths.
\end{subsection}% Setup your environment variables

\begin{subsection}{Create a tmp directory}
Then, create a directory called 'tmp' at the root of each harddrive, i.e.:
\begin{verbatim}
C:\tmp
D:\tmp
...
\end{verbatim}
You need at least a 'tmp' dir at the drive your CDKDIR is located at and possibly for the Visual C++ 6.0 installation drive.
\end{subsection}% Create a tmp directory

\begin{subsection}{Reboot}
Finally, you'll have to reboot the system.
\end{subsection}% Reboot

\begin{subsection}{Obtain latest ALL patch from CtP2 Source Code Project}
Go to the \href{http://apolyton.net/forums/showthread.php?threadid=100609&goto=lastpost}{PROJECT: Altered source files} thread, and look for the latest post with a .zip file containing the string all, e.g. \href{http://apolyton.net/csd.php?http://apolyton.net/go.php?http://page.mi.fu-berlin.de/~guehmann/CTP2/2004.09.13.CTP2.All.zip}{2004.09.13.CTP2.All.zip}. Unpack that file into your ctp2 source code directory, e.g. \textit{C:\textbackslash{}ctp2source}.
\end{subsection}% Obtain latest ALL patch from CtP2 Source Code Project

\begin{subsection}{Optional: Get up to date until latest post}
Beginning from the post where you found the latest ALL patch, descend to the latest post and perform the following steps in post order:

\begin{enumerate}
\item{Download posted files}
\item{For each posted file, do the following:}
\item{If posted file is a .zip or .rar archive, and there are no instructions,
unzip it to your ctp2 source code directory.\\
Perhaps you get warnings by your zip-Software regarding files to be overwritten. This is ok, so let your packer replace your local files by the newer ones in the archives you downloaded.}
\item{If the previous didn't apply, follow the instructions of the post where you got the file from to apply the changes to your ctp2 source code directory.}
\end{enumerate}
\end{subsection}% Optional: Get up to date until latest post

Alternatively, update from the SVN respiritory.

\begin{subsection}{Make a backup of the clean source directory}
Backup the sources to another directory. It will help you playing with the sources, later.
\end{subsection}% Make a backup of the clean source directory

\end{section}% Windows

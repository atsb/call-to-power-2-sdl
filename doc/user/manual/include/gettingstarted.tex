\chapter{Getting Started}
\label{chap:gettingStarted}

In this section you will get a guided tour through a few dozen years/turns,
to help familiarise you to various parts of CtP2.

\section{Quick start}

If you want to learn about the options available to you before you start the
game, then read the next section, \vref{sec:pre-gameSetup}.

If you don't want to bother with pre-game setup then you can easily begin a
game quickly by using default setup --- so don't go into the `Options' menu yet.
First let's start a `New Game'.

You should see now see the `New Game' setup screen. For this game we will leave
the options alone. You will be Julius Caesar for this walk-through. Once you've
had a look at the various option buttons, click the green `Launch' button at
the bottom of this screen.  Now you can skip the next section and go straight
into learning how to play at section \vref{sec:turns}.

\section{Pre-game Setup}
\label{sec:pre-gameSetup}

After the intro movie has played you will find yourself at the Apolyton Edition
start menu.

From here you can select the type of game
you wish to play and access various game
options.

On the left hand side we have the multi-player options, these are:

\begin{description}
\item[Network game] will allow you to setup a
multi-player on-line game or a LAN game.
\item[Email game] will allow you to play a game by email (known as a PBEM game).
\item[Hotseat] will allow you to play a Multi-Player game, with all players on
one PC.
\end{description}

On the right hand side of the screen we have the single player options, these
are:

\begin{description}
\item[New Game] starts the single player default game.
\item[Load Game] loads a previous game.
\item[Tutorial] starts the in-game tutorial, a good place for new players.
\end{description}

At the bottom of the screen we have three other buttons, going from left to
right these are:

\begin{description}
\item[Options] allows you to setup some game parameters and sound and graphics
settings.
\item[Credits] lists all the people who have
helped in the Apolyton Edition remake.
\item[Quit] don't do it! 
\end{description}

\section{Turns}
\label{sec:turns}
What is a turn, what happens between turns, when a new turn starts?

\section{Fog of War}

What's all that black stuff, why are some parts of the visible map `greyed
out'?

\section{Settling}

What to do with that donkey guy, what to pay attention to when finding sites to
settle?

\section{The User Interface}

After a short loading screen, you will find that you have two settler units on
a small piece of visible land (The number 2 in the unit `shield' indicates two
units in the stack). Most of the game map is completely black.
Before we move our settler units lets have a look around at what does what in
the User Interface (UI).

\subsection{Mini-map}
At the bottom left we have a large grey box --- this is the mini map, and gives
you a larger view of the world around you. At this moment it will be nearly all
black, as you know nothing of the world around you yet. You can click in the
mini-map to move the game view to that area of the map. At the top of the
mini-map box are various filter options for the amount of detail you wish to
display on the mini-map. Once you are further into the game these can help you
with an at-a-glance worldview on how your empire is doing compared to the other
players. Until you have explored a decent amount of the map around you, we can
minimise the mini-map to give us a better view of the game world. Click the
minimize button in the top right of the mini-map now, you can always expand it
later as and when you need it.

\subsection{Menu bar}
\index{menus}
At the top of the screen is a grey bar with many options. From left to right
they are as follows.  Keyboard shortcuts are given in parentheses.

\begin{description}
\item[Empire] provides links to:
\index{menus!empire}
\begin{itemize}
\item Empire Manager screen (F1)
\item Trade --- Market screen (F4)
\item Trade --- Summary screen (Ctrl+t)
\item Science Manager screen (F6)
\item Gaia Controller screen (Ctrl+g)
\end{itemize}
\item[Cities] provides links to:
\index{menus!cities}
\begin{itemize}
\item Build Manager screen (Ctrl+b)
\item City Manager screen (F3)
\item National Manager screen (F2)
\end{itemize}
\item[Units] provides links to:
\begin{itemize}
\item Unit Manager screen (F8)
\item Army Manager screen (.)
\end{itemize}
\item[Diplomacy] provides links to:
\begin{itemize}
\item Diplomacy Manager screen (F7)
\item New Proposal screen (Ctrl+d)
\end{itemize}
\item[Status] provides links to:
\begin{itemize}
\item Great Library screen (F5)
\item Ranking (world) screen (Ctrl+k)
\item Score screen (F9)
\item Wonders screen (W)
\end{itemize}
\item[Options] accesses various in game options that effect different things:
\begin{itemize}
\item Zoom in on the world map (-)
\item Zoom Out on the world map (+)
\item Gameplay (F)
\item Graphics (Ctrl+f)
\item Sound (Ctrl+v)
\item Music (Ctrl+m)
\item Advanced (Ctrl+p)
\item Cheat Mode (!)
\item Save Game (S)
\item Load Game (L)
\item Restart (Ctrl+z)
\item New Game (Ctrl+x)
\item Quit (Q)
\end{itemize}
\end{description}

\subsection{Main user interface hub}

In the bottom right of the screen you will find the main user interface hub.

This is the main UI console that you will use to access CtP2's various
management screens.

The circular part contains buttons which provide access to
screens described in the following sections.

\subsubsection{Empire Manager}
The Empire Manager screen is
divided into two pages:

The first page is Domestic Policy, on which
you can set the amount of food your people receive.
The work hours per day they work and the amount of this overall production that
goes into public works (PW).

Also you can set the amount of gold they get paid as wages and the amount of
the overall gold that goes into your scientific research.

The second page is Government, on which you
can get information about the efficiency of the
government type your empire is working under, and a pull down list of the types
of governments you can currently choose.

At the beginning you have only one choice
--- Tyranny, not a great system and you
should aim to move to a more advanced one as soon as you can by researching the
required techs (see Science Manager).

\subsubsection{Unit Manager}
The Unit
Manager screen gives you
information on all your units and is divided into two pages:

Unit Statistics, on which you can see a complete list of all your unit types
and their combat ratings. Also on the left is a small box in which your
Military Advisor gives you information, you can toggle this box on/off with the
`Advisor' button on the bottom left of the Unit Statistics screen.
Just above the `Advisor' button is the unit war status slider --- this can be
used to change the war readiness of your units. It has three settings:

\begin{description}
\item[At peace] your units cost little to support and are at one-third combat
strength.
\item[On alert] your units cost a medium amount
and are at half combat strength.
\item[At war] your units cost the most in support and
are at full combat strength.
\end{description}

You start with `At war' support costs, and it's best to leave it like that for
now.

Note that it takes a few turns (exactly how many depends
on your government) for your War Status to change to the new
selection, so take this into account as you can leave yourself vulnerable if
you make a change at the wrong time!

Just below the `War Status' information panel, is the `Disband Unit' button ---
be careful with this button as its possible to disband all units of a certain
type when in our current `Unit Statistics' screen!

The last button on the bottom right is the `Quit' screen button; this closes
the `Unit Manager' screen.

Tactical Info, which gives you a complete breakdown of each and every
unit. It displays the units name, army number, any orders, its location and its
current health rating.
From here you can highlight each individual unit in your armies --- if I need to
use the `Disband Unit' function, I do it from this screen rather than the `Unit
Statistics' page.

\subsubsection{Diplomacy Manager}
This screen allows you
to conduct diplomatic relations with
any of the other Empires you have come into contact with. Next to the nations
name this screen displays the nations regard of you, its relative strength,
whether you have an embassy established, and the type of alliances or treaties
you have with each other.
To engage in diplomacy you need to select a nation from the list and select one
of the options at the bottom of the screen. These are, from left to right:

\begin{description}
\item[Intelligence] provides more detail on
that particular empire and its relations
with you, using the pages called foreign relations, Domestic and Science.

\item[Create proposal] initiate diplomatic proposals in which you can set the
demands, your tone of address, and various other diplomatic options. There are
many options available to you. The basic process involves these few steps:

\begin{enumerate}
\item Select the tone of your proposal (from kind to angry)
\item \label{step:select} Select either a Request, Offer or Treaty
\item Optionally select an additional Request,
Offer or Treaty to make an exchange with
that selected in step \ref{step:select}.
\item Send it off
\end{enumerate}

You will need to have encountered another civilisation before you can undertake
diplomatic actions. It's worth experimenting with all the options to get the
hang of it (you can always go back some steps if you make a mistake).

\item[Declare War] for when diplomacy breaks down.

\item[Embargo] your Empire will automatically
close all trade links with the chosen
party.
\end{description}

\subsubsection{Science Manager}
This screen gives you information on the current advance you
are researching in the top left and centre graphic box. The number below the
picture is the amount of turns remaining before you have learnt this new
advance. The bottom half of the screen is used to display the advances you have
already learnt, represented by your civilisations colour in the first column of
blocks. There are seven other blocks running across the screen.

As you meet other civilisations they will have their colours represented here.
It helps you to see how far ahead or behind you are in your advances in
comparison to the other civilisations.

Clicking on either the `Change Research' button, or the number of turns
remaining box under the picture, will enable you to change the advance you want
to research.

Its one of the first thing I do when starting a game as often you will be on a
difficult advance to discover as a default.

A useful function in the `Change Research' screen, is the ability to set
advance goals. At the bottom of the screen is a `Goal' button. Clicking on this
enables you to access the `Great Library' and select a particular advance goal
for you to aim for. When you do this all the subsequent advances that lead to
this goal will have an asterix (\texttt{*})
symbol next to them, so you can see what
advance choices you make will lead you quickly to your chosen Goal, very handy.

At the start of the game I usually set my Goal to `Monarchy' --- its good to get
out of `Tyranny' as soon as you can, to have a more productive level of
government.

\subsubsection{The Great Library}
If you want some information about any in game aspect of
Call To Power 2, this is the place to go.
Down the left of the screen are ten buttons that will give you lists of all
game:

\begin{enumerate}
\item Advances
\item Units
\item City Improvements
\item Wonders
\item Terrain
\item Goods
\item Tile Improvements
\item Governments
\item Unit Orders
\item Concepts
\end{enumerate}

Everything is listed in alphabetical order and there is even a search function
at the top left of the screen. At the bottom of the screen you can access the
`Set Goal' options as discussed above.

One word of warning --- spending too long in the `Great Library' can spoil some
of the surprise/mystery of the game, so if you are new to the game use it
sparingly.

\subsubsection{Trade Manager}
Use this screen to setup trade routes when you have `Goods' to
trade. There are two pages in the `Trade Manager' screen:

\begin{description}
\item[Market] lists all the available trade routes you can setup, if
you have enough trade `caravans' to do so. Each trade route has a cost in the
amount trade `caravans' required to set the route up. You need to build
`caravans' in your cities when you have researched the required advance. Once
you have selected the route you want to open by clicking on it, and have enough
trade `caravans', click `Create Route' at the bottom of the screen.

The direction the route travels will be decided for you. Sometimes if this
route goes through enemy territory it can be a good idea to think about using
another route, or you will find it getting pirated fairly often!

At the top of the screen are three buttons that toggle the trade routes you can
choose:

\begin{description}
\item[Own] will only display the routes you can setup within your own empire.

\item[Friendly] displays all your routes plus all those of other friendly
civilisations.

\item[All] all possible trade routes are displayed.
\end{description}

Next to these buttons is the `Cities per good' slider, this can be set to
between 1 and 5. What this does is(..?)

At the bottom left of the screen is another `Advice' button, this gives access
to the trade advisor whom can give you suggestions and also displays
information on the trade routes and caravans you have or have used.

\item[Trade Summary] shows you information on all your active trade
routes. It also will enable you to see if any of your routes are being pirated,
a pirate flag will appear to let you know. If this happens it is a good idea to
follow the trade route with a unit if possible and remove the pirate (or when
you can see which civilisation the pirate belongs too, use the diplomacy screen
to tell them to stop!).

The amount of money the trade route creates and the amount of caravans needed
to keep it open are also displayed here.

At the bottom of the screen there is a `Disband Route' button, this enables you
to cancel any route you select from the list displayed.
\end{description}

\subsubsection{National Manger}
This gives you a quick overview of all your cities. There are
three main pages of information within this screen.

\begin{description}
\item[Resources] on this screen you get a resource breakdown for each city, it
displays the city name, population size, happiness, food, production, gold,
science, pollution and crime rate. You can toggle what priority this
information is displayed in by clicking on the title tab (if you want to know
which cities produce most food for example, click on the loaf of bread and it
will toggle between the highest and lowest food producing city).

\item[Status] this screen enables you see what
each city is producing at that time
and how many turns it will take to finish. Also at the bottom of the screen you
have options to toggle on/off the city `Mayor' function and set parameters for
it.

To turn the Mayor on, select a city from the list then click the `Mayor' button
in the bottom left of this screen. The information panel above will update any
changes you make with the Mayor status. There is a pull down menu next to the
mayor button where you can select a priority for the mayor to follow (for
example `production').

To be honest I never use Mayors, they are
pretty hopeless at their jobs! I prefer
to manage all my cities manually --- still if you want a laugh/cry, then give
them a go.

Just to the right of the `Mayor' button is the `Rush Buy' button. When you have
selected a city from the list in the
`Status' screen you will be able to quickly
build whatever is in that cities build queue at the time, by clicking the `Rush
Buy' button. The cost is in gold and is displayed to the right of the button.
This can be useful when in a Wonders race, or when you notice an enemy army
approaching a poorly defended city.

\item[Specialists]

\item[City Manager] This screen gives you detailed information on a specific
city.
\end{description}

At the bottom of the screen are two buttons that can be selected when a city in
the list is highlighted.

\begin{description}
\item[Build Manager] takes you into the build queue for the selected city.
\item[Disband] will disband any selected city, so be careful!
\end{description}

\section{Units}
How to build units, how to move them around, how to attack? Basic info on
different types of units.

\section{Cities}
What�s their purpose, how to build things, what are specialists and mayors?

\section{Terrain \& PW}
What impact do they have on resource collection, movement, defense?

\section{Empire}
What are empire settings and governments?

\section{Science}
What the purpose of research?

\section{Goods \& Trade}
What are good and how to trade them?

\section{Opponents, War \& Diplomacy}
What are those other guys doing there, how to I fight them, how do I stop
fighting them?

\section{Winning the Game}
What to do to win, what victory types are there?

